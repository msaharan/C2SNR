\documentclass[12pt,a4paper]{article}
\usepackage[utf8]{inputenc}
\usepackage[T1]{fontenc}
\usepackage{amsmath}
\usepackage{amsfonts}
\usepackage{amssymb}
\usepackage{graphicx}
\begin{document}
\section{Equation 3 of the paper}
\begin{eqnarray}
\Delta^2_k(\ell) = \frac{9 \Omega_m^2 H_0^4 \pi }{4} \frac{1}{\ell} \int d\chi' \chi' l\left[ \int_\chi^\infty d\chi_s p(\chi_s) (1 - \frac{\chi}{\chi_s})   (1 + z) \right]^2  \Delta_m^2(k = \ell/\chi, z)
\end{eqnarray}

Equation 2.69 of Bartlenn and Schneider
The redshift distribution of the faint blue galaxies 
\begin{eqnarray}
p(z) dz = \frac{\beta}{z_0^3 \Gamma(3/\beta)} z^2 e^{-(z/z_0)^\beta} dz
\end{eqnarray}
This expression is normalised to $ 0 \leq z <  \infty $ and provides a good fit to the observed redshift distribution (e.g. Smail et al.  1995b). The mean redshift hzi is proportional to $ z_0 $, and the parameter $\beta$ describes how steeply the distribution falls off beyond $ z_0 $. For $\beta = 1.5$, $\langle z\rangle \sim 1.5 z_0 $. The parameter $ z_0 $ depends on the magnitude cutoff and
the colour selection of the galaxy sample.
\end{document}